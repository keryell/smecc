% -*-coding: latin-9;-*-

%% To deal with the colors and number of slides per page according to the
%% \VersionPapier switch :
\ifx\VersionExpose\UnTrucInexistant%
% Version �l�ve
\documentclass[handout,compress,10pt,hyperref={hyperindex},xcolor={svgnames,x11names}]{beamer}
\usepackage{pgfpages}
\pgfpagesuselayout{4 on 1}[a4paper,landscape,border shrink=5mm]
\else
% Version enseignant
\documentclass[compress,10pt,hyperref={hyperindex},xcolor={svgnames,x11names}]{beamer}
\fi
%\documentclass[11pt,a4]{powersem}
\usepackage[latin9]{inputenc}

\usepackage[T1]{fontenc}
\usepackage{amsmath,wasysym,marvosym,soul}
\usepackage{multicol}
\usepackage{tikz}
\usetikzlibrary{positioning}

\input{macros_jpl94}

% To draw diagonal line on some text:
\usepackage[thicklines]{cancel}
\renewcommand{\CancelColor}{\red}

\usepackage[hyper,procnames]{listings}
\usepackage{alltt}%verbatim,
%\usepackage[linkbordercolor={1 1 1},bookmarks=true,colorlinks=false,urlbordercolor={1 1 1},runbordercolor={1 1 1},citebordercolor={1 1 1},bookmarksnumbered=true,hypertexnames=false,hyperindex]{hyperref}
%usepackage{tabularx}

%\usepackage{beamer_rk}
\usepackage[LeftSideBar,ShadedBackground]{beamer_rk}
%\usepackage[WhiteBackground]{beamer_rk}


\usepackage{perso2e,bbold2e,getVersion,exemple}
%\usepackage{beamer_rk,perso2e,bbold2e,LienPDF,example}
\usepackage{makeidx}
%\usepackage{multicol,tabularx}
\usepackage{pst-node}

\usepackage[greek,francais,english]{babel}
\frenchbsetup{og=�, fg=�, ThinColonSpace=true,
  % ShowOptions,
  % �vite de casser les r�f�rences bibliographiques du style harvard avec
  % des ":" dedans :
  %AutoSpacePunctuation=false
}
% S'arrange pour avoir une virgule qui ne rajoute pas de blanc dans les
% nombres � virgule :
%\iffalse
\DecimalMathComma
% M�me pas besoin pour les ligatures � 2 accents en fait :
%\languageattribute{greek}{polutoniko}

% Et pour avoir \nombre qui marche :
\usepackage[autolanguage]{numprint}

\lstset{extendedchars=true, language=C++, basicstyle=\scriptsize\ttfamily, numbers=left,
  numberstyle=\tiny, stepnumber=2, numberfirstline=true,
  tabsize=8, tab=\rightarrowfill, keywordstyle=\color{orange}\bf,
  stringstyle=\rmfamily, commentstyle=\rmfamily\itshape}
% Un peu violent :
%  index=[1][keywords],indexprocnames=true%\lstloadlanguages{Basic}

%\usepackage{beamerthemesplit}
%{\usebeamercolor{palette primary}}
%\setbeamertemplate{sidebar canvas}[vertical shading][top=palette primary.bg,middle=white,bottom=palette primary.bg]

\makeindex

\getRCSDollarsVersion$Revision: 185 $

\usepackage[dvips]{preview}
\PreviewEnvironment{trans}
\PreviewEnvironment{frame}

\newcommand{\FigCentreTourne}[1]{\centerline{\includegraphics[width=\vsize,angle=-90]{#1}}}
%\newcommand{\boiteblanc}[1]{\psframebox*[fillcolor=papayawhip]{#1}}
\newcommand{\boiteblanc}[1]{\psframebox*[fillcolor=NavajoWhite1]{#1}}

%\DealWithTwoUp

%\newcommand{\reddrawline}{{\red \noindent\leavevmode\hrulefill}}

\newcommand{\Idee}{{\red\Handwash}\xspace}
%\Stopsign \Laserbeam \Estatically \Attention \Idee
%\FilledRainCloud\Coffeecup

\newcommand{\Cnn}{C$_{99}$\xspace}

%% Plan plus petit :
%\renewcommand{\OutlineFontSize}{\scriptsize}
% Plan en une colonne :
\renewcommand{\OutlineColumnNumbers}{1}
\renewcommand\outlinename{Outline}


\usetheme{gtc}
\usecolortheme{gtc}


\newcommand{\GrosRouge}[1]{{\red #1}}

%\logo{\href{http://hpc-project.com}{\pgfuseimage{logo-HPC-Project}}\href{http://www.par4all.org}{\pgfuseimage{logo-Par4All}}}
\pgfdeclareimage[width=0.8cm]{logo-HPC-Project}{Logo_HPC-Project-512x247-crop}
%\pgfdeclareimage[width=0.75cm]{logo-Wild-Systems}{WildSystems-site-logo}
\pgfdeclareimage[width=0.8cm]{logo-PIPS}{Logos/PIPS/logo-pips.small}
\pgfdeclareimage[width=0.8cm]{logo-Par4All}{Par4All-logo}
%\logo{\hbox to 1.5cm{\vbox{\href{http://pips4u.org}{\pgfuseimage{logo-PIPS}}\vss}}}
\logo{\hbox to 0.8cm{\vbox{\href{http://pips4u.org}{\pgfuseimage{logo-PIPS}}\\\href{http://www.par4all.org}{\pgfuseimage{logo-Par4All}}\\\href{http://hpc-project.com}{\pgfuseimage{logo-HPC-Project}}\vss}}}
%\logo{\hbox to 0.75cm{\vbox{\href{http://wild-systems.com}{\pgfuseimage{logo-Wild-Systems}}\\\href{http://www.par4all.org}{\pgfuseimage{logo-Par4All}}\vss}}}
%\logo{\hbox to 0.75cm{\vbox{\pgfuseimage{logo-HPC-Project}\\\pgfuseimage{logo-telecom-bretagne}\\\pgfuseimage{logo-HPCAS}}\hss}}

\newcommand{\HPCIN}{%\protect\usebeamercolor{normal text}
  \textcolor{red}{\st{\textsc{hpc}}}plain computing\xspace}

\title{SMECY Internal Representation\\
  ---\\
  C + \texttt{\#pragma}\\
  ---\\
  SMECY-C or SME-C?
}

\date{2011/05/25\\
  ---\\
  SMECY General Assembly\\
  Delft}

\author[Ronan \textsc{Keryell} \texttt{et al.}]{Michel
  \textsc{Barr�teau}\inst{2} \and R�mi \textsc{Barr�re}\inst{2} \and Ronan
  \textsc{Keryell}\inst{1}}

\newcommand{\HPCP}[0]{\href{http://hpc-project.com}{HPC Project}}

\newcommand{\TB}[0]{\href{http://hpcas.enstb.org}{Institut T�L�COM/T�L�COM
    Bretagne/HPCAS}}

\newcommand{\CRI}[0]{\href{http://www.cri.mines-paristech.fr}{Mines ParisTech/CRI}}

\newcommand{\TRT}[0]{\href{http://www.trt.thalesgroup.com}{THALES Research
    \& Technology}}

\institute[SMECY GA --- 2011/05/25 @ Delft]{\inst{1}\HPCP\\\inst{2}\TRT}

\begin{document}

%\DealWithColors

\begin{frame}[plain]
  \titlepage
\end{frame}

%\slidepagestyle{CRI}

%\LicenseTrans

%\AtBeginSection[]
%{
%}
%\AtBeginSubsection[]
%{}



% Here because of catcode wizardy I guess :-(
% Should go into tikz_hpc
% Could use \pgftext instead of using "text depth=0pt"
\newcommand{\PlaceTextNode}[2]{\tikz{\node[inner sep=0pt,text depth=0pt] (#1) {#2};}}




%\input{Lib/Algorithms/Parallel_Prefix/parallel_prefix}

%\end{document}

\begin{frame}[fragile]{SMECY C programming environment}
  \begin{itemize}
  \item Focus classical programming with legacy applications
  \item Close to classical sequential C with C unified memory model
  \item Add some \verb|#pragma| to specify SMECY details
  \item Use cases
    \begin{itemize}
    \item Direct high-level programming
    \item System high-level synthesis
      \begin{enumerate}
      \item Plain C99 or Matlab or SPEAR-DE or Fortran or DSL or Ptolemy
        II or...
      \item Tool: analyze and parallelize the code by adding automatically
        parallel and mapping pragma
      \item SMECY C
      \end{enumerate}
    \item Hardware high-level synthesis
      \begin{enumerate}
      \item C99 program with SMECY pragma
      \item SMECY compiler with target description + target API
      \item Executable on SMECY target with host and accelerator parts
      \end{enumerate}
    \end{itemize}
  \end{itemize}
\end{frame}


\begin{trans}{Pragma tour}
  \begin{itemize}
  \item High Performance Fortran HPF (data parallelism, memory
    distribution, data remapping...)
  \item OpenMP 3.1 (data and task parallelism...)
  \item BlueBee (parallelism \& hardware mapping)
  \item hArtes
  \item Many others
    \begin{itemize}
    \item Lot in hardware synthesis world
    \item \Attention Bibliography to finish... Help! Already done by a
      SMECY partner?
    \end{itemize}
  \end{itemize}
\end{trans}


\begin{frame}{Sequential equivalence}
  \begin{BoiteB}{Same semantics}
    \begin{center}
      Sequential $\equiv$ Parallel $\equiv$ SMECY
    \end{center}
  \end{BoiteB}
  \begin{itemize}
  \item Do not perturb the programmer... \Attention
    \begin{itemize}
    \item Sequential execution gives same results as any SMECY
      target implementation
    \item Functional simulator for free! \smiley{} (think as SystemC...)
    \item Easy debug of applications (do not even need of SMECY tools or
      hardware)
    \item Easy debug of SMECY tools too... \smiley

      \pause

    \item Can test the concepts before building tools!!!\\
      \vavers Do not sequentialize the project! \smiley
    \end{itemize}
  \item OpenMP execution
    \begin{itemize}
    \item Parallelized version of functional simulator
    \item Debug parallelized version of code
    \item Only need OpenMP compiler + SMP machine
    \item Can run Hellgrind or other execution verifiers \smiley
    \end{itemize}
  \end{itemize}
\end{frame}


\begin{frame}[fragile]{SMECY programming model}
  \begin{itemize}
  \item Sequential C programming model
  \item Unified classical C shared memory model
  \item OpenMP possible on SMP host and accelerators if available in SMECY
    target
  \item Some functions can be executed on some hardware accelerators or
    other processing elements
  \item No explicit communication between different target memory spaces
  \item Help compiler with pragma or API to deal with
    \begin{itemize}
    \item Parallel execution
    \item Mapping to specific hardware or processing elements
    \item Consumed and produced data
    \item Data remapping to cope with hardware constraints
    \item Asynchronism \& synchronization on hardware resources
    \end{itemize}
    Use specific naming space to avoid conflicts with other existing
    pragma
\begin{lstlisting}
#pragma smecy ...
\end{lstlisting}
  \end{itemize}
\end{frame}


\begin{trans}{Parallel execution}
  \begin{itemize}
  \item Use OpenMP 3.1 \verb|#pragma| syntax \& API
  \item $\exists$ OpenMP API reference implementation for sequential
    execution \smiley
    \begin{itemize}
    \item For example \lstinline|omp_get_num_procs()| return always 1
    \end{itemize}
  \end{itemize}
\begin{lstlisting}
#pragma omp parallel for
  for (int i = 0; i < size; i++)
    out [i] = in [i] + 1;

#pragma omp parallel sections
  {
    {
      Add(200*2, (int *) tab, (int *) tab);
    }
#pragma omp section
    {
      Add(200*2, &tab[2][0], &tab[2][0]);
    }
#pragma omp section
    {
      Add(200*2, &tab[4][0], &tab[4][0]);
    }
  }

#pragma omp parallel
{
#pragma omp task
  this_may_be_in_another_task();
}
\end{lstlisting}
\end{trans}


\begin{frame}[fragile]{Hardware mapping}
  \begin{itemize}
  \item Specify where a function is executed
  \item Use target-specific identifiers
  \end{itemize}
\begin{lstlisting}
#pragma smecy map(GPP, 0) ...
  bool result = Test(200*6, (int *) tab);

#pragma smecy map(PE, 4) ...
      Add(200*2, &tab[4][0], &tab[4][0]);
\end{lstlisting}
\end{frame}


\begin{trans}{Data flow information}
  \begin{itemize}
  \item A SMECY compiler needs to add communications around accelerator
    calls
  \item Difficult in the general case for a compiler to track information
    flow
  \item \vavers Use annotation to describe memory use-def
  \end{itemize}
\begin{lstlisting}
void Gen(int *out, int size) {
  for (int i = 0; i < size; i++)
    out [i] = 0;
}
    [...]
#pragma smecy map(GPP, 0) arg(1, [6][200], out)
  Gen((int *) tab, 200*6);
    [...]
void Add(int size, int in[size], int out[size]) {
  for (int i = 0; i < size; i++)
    out [i] = in [i] + 1;
}
    [...]
#pragma smecy map(PE, 4) arg(2, [2][200], in) arg(3, [2][200], out)
      Add(200*2, &tab[4][0], &tab[4][0]);
\end{lstlisting}

  \break

  \begin{itemize}
  \item Sparse sub-array access
  \end{itemize}
  \includegraphics[width=\hsize]{examples/2D_example-output}
\begin{lstlisting}
#pragma smecy map(PE, 0) arg(3, inout, [HEIGHT][WIDTH]			\
			     /[HEIGHT/3:HEIGHT/3 + HEIGHT/2 - 1]	\
			     [WIDTH/8:WIDTH/8 + HEIGHT/2 - 1])
    square_symmetry(WIDTH, HEIGHT, image, HEIGHT/2, WIDTH/8, HEIGHT/3);
\end{lstlisting}
\end{trans}


\begin{trans}{Data remapping}
  \begin{itemize}
  \item Some (hardware) functions need to access memory in a specific pattern
    \begin{itemize}
    \item Vector operation on a part of a 2D or 3D array...
    \end{itemize}
  \item Need to adapt memory layout between use and function requirements
  \item Because of sequential equivalence, even the sequential code has
    this issue
    \begin{itemize}
    \item \vavers Use an API instead of a \verb|#pragma|
    \item \vavers Provide SMECY API for non-SMECY target (sequential,
      OpenMP)
    \end{itemize}
  \item Example
    \begin{itemize}
    \item \lstinline|invert_vector()| operates on continuous memory
    \item Can be applied on continuous memory (horizontal line in an image)
    \item ... or on discontinuous memory \frownie{} (vertical line in an
      image)
    \end{itemize}
  \end{itemize}
  \includegraphics[width=\hsize]{examples/remapping_example-output}
\begin{lstlisting}
  // Draw 70 horizontal lines and map operation on 8 PEs:
#pragma omp parallel for num_threads(8)
  for(int proc = 0; proc < 70; proc++)
    // Each iteration is on a different PE in parallel:
#pragma smecy map(PE, proc & 7)			\
              arg(2, in, [1][LINE_SIZE])	\
              arg(3, out, [1][LINE_SIZE])
    // Invert an horizontal line:
    invert_vector(LINE_SIZE,
		  &image[HEIGHT - 20 - proc][WIDTH/2 + 2*proc],
		  &image[HEIGHT - 20 - proc][WIDTH/2 + 2*proc]);
\end{lstlisting}

  \break

\begin{lstlisting}
  /* Here we guess we have 5 hardware accelerators and we launch
     operations on them: */
#pragma omp parallel for num_threads(5)
  for(int proc = 0; proc < 5; proc++) {
    /* This is need to express the fact that our accelerator only accept
       continuous data but we want apply them on non contiguous data in
       the array */
    int input_line[LINE_SIZE];
    int output_line[LINE_SIZE];
    /* We need to remap data in the good shape. The compiler should use
       the remapping information to generate DMA transfer for example and
       remove input_line array */
    SMECY_remap_int2D_to_int1D(HEIGHT, WIDTH, HEIGHT/3, 30 + 20*proc,
			       LINE_SIZE, 1, image,
			       LINE_SIZE, input_line);
    // Each iteration is on a different PE in parallel:
#pragma smecy map(PE, proc) arg(2, in, [LINE_SIZE]) arg(3, out, [LINE_SIZE])
    invert_vector(LINE_SIZE, input_line, output_line);
    SMECY_remap_int1D_to_int2D(LINE_SIZE, output_line,
			       HEIGHT, WIDTH, HEIGHT/3, 30 + 20*proc,
			       LINE_SIZE, 1, image);
  }
\end{lstlisting}
\end{trans}


\begin{frame}[fragile]{Synchronization}
  \begin{itemize}
  \item By default, synchronous function calls to accelerators
  \item Asynchronous execution needs OpenMP threads around accelerator
    calls
    \begin{itemize}
    \item May be overkill if a lot of fine grain accelerator calls to do
      pipelining...
    \end{itemize}
  \item \vavers Introduce asynchronous function calls
\begin{lstlisting}
#pragma smecy map(...) async
\end{lstlisting}
  \item Rely on synchronization \verb|#pragma|
\begin{lstlisting}
#pragma smecy wait(PE,2)
\end{lstlisting}
  \item Syntax/concept still to finalize with an example of  pipelined
    application...
  \end{itemize}
\end{frame}


\begin{frame}[fragile]{Compilation}
  \begin{itemize}
  \item Lot of information in \verb|#pragma| and API
  \item Simple use-def analysis
  \item Simple geometrical array analysis to generate communications
  \item No need for polyhedral model
  \item Recycle some concepts from:\\
    Corinne ANCOURT, Fabien COELHO, Fran�ois IRIGOIN and Ronan KERYELL. �
    A Linear Algebra Framework for Static HPF Code Distribution. � in
    \emph{CPC'93 : Fourth Workshop on Compilers for Parallel Computers.}
    \textbf{Delft, Netherlands, December 1993.} \smiley
  \end{itemize}
\end{frame}


\begin{frame}[fragile]{Conclusion}
  \begin{itemize}
  \item Classical C programming and other languages
  \item \verb|#pragma|tic approach
  \item Simple \verb|#pragma| \& API instead of specific DSL to learn
  \item No need to define explicit communications
  \item Can be used to program SMECY applications
  \item Usable as \emph{a part of the} Internal Representation between
    SMECY tools
  \item Should be easy to compile
  \item Sequential equivalence semantics for easy programming and
    debugging of applications, tools, with or \emph{without} SMECY
    compilers and targets
    \begin{itemize}
    \item \vavers Few small already examples available and run in
      sequential and with OpenMP
    \item Need to port use-case applications or to generate SMECY C with
      automatic tools
    \end{itemize}
  \item Syntax detail of \verb|#pragma| \& API still to tweak and
    contribute!
  \item Do we need higher level \verb|#pragma| (\emph{pipeline this
      loop}...)? Different levels? Different tools?
  \end{itemize}
\end{frame}


\section*{Table of content}

\begin{multicols}{2}
  \tiny
  \tableofcontents[frametitles]
  \textbf{You are here!}\hfill\expandafter\the\csname c@page\endcsname
\end{multicols}

\end{document}

%%% Local Variables:
%%% mode: latex
%%% ispell-local-dictionary: "american"
%%% End:
