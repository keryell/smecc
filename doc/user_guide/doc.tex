%============================%
%                            %
%     DOC SMECY COMPILER     %
%                            %
%============================%

%=====HEADER=====%
\documentclass[a4paper,11pt]{article}
\usepackage[utf8]{inputenc}
\usepackage[T1]{fontenc}
\usepackage[english]{babel}
\usepackage{hyperref}
%\usepackage{algorithmic}
%\usepackage{algorithm}
%\usepackage[left=90px,hmarginratio=1:1,top=25mm]{geometry}
%\usepackage{amsthm}
%\usepackage{amsmath}
%\usepackage{amssymb}
%\usepackage{mathrsfs}
\usepackage{graphicx}
%\renewcommand{\algorithmiccomment}[1]{//#1}
\newcommand{\compiler}{SMECC}
\newcommand{\scompiler}{SMECC }

%=====TITLE=====%
\title{\scompiler -- A SME-C compiler using ROSE}
\author{Vincent \sc Lanore}

%=====DOCUMENT=====%
\begin{document}
	\maketitle

	\section{Presentation}
		\scompiler (for SME-C Compiler) is a C99/C++ compiler able to process SMECY pragmas to map function calls to hardware accelerators. \scompiler is written using ROSE \cite{usermanual,tuto}, a tool to write source-to-source translators. First, input code is parsed using ROSE front-end (depends on the input language) into a SageIII AST (ROSE's AST). Then, SMECY pragmas are processed and translated into calls to the SMECY API. Finally, the ROSE backend is called to produce C code with calls to the SMECY API which can be compiled using a regular compiler.

	\section{Features}
		\scompiler currently supports the following features :
		\begin{itemize}
			\item translation of \verb+#pragma smecy map+ directives applied to function calls of the form \verb+function(parameters);+, \verb+varName = function(parameters);+ or \verb+type varName = function(parameters);+;
			\item support from following \verb+arg+ clauses : type (in, out...), size and range;
			\item verification of the contiguity of the vector arguments in memory;
			\item computing ranges to get the actual dimension of any argument, printing warning when arguments with dimension $>1$ are used as vectors;
			\item automatically finding the size of arrays if not specified in pragma.
		\end{itemize}

	\section{How to use}
		\paragraph{Environment} Before using the compiler a few environment variable should be set.
		\begin{itemize}
			\item add \scompiler directory to the \verb+$PATH+ :
\begin{verbatim}
export PATH=smecc_directory/:$PATH
\end{verbatim}
			\item set \verb+SMECY_LIB+ to the directory containing the
              SMECY library :
\begin{verbatim}
export SMECY_LIB=smecy_lib_directory/
\end{verbatim}
            \end{itemize}

		\paragraph{Usage}
		\scompiler works mostly like a regular C/C++ compiler. Most C/C++ usual compiler flags will work with a few exceptions and additions (see below). By default, it will \emph{not} compile smecy pragmas (see below).

		\paragraph{Specific flags}
		\scompiler supports some specific flags. Here are a few examples, for a more complete list type \verb+smecc --help+.
		\begin{itemize}
        \item \verb+-smecy+ triggers smecy pragmas
          translation/compilation; if pragmas contain many expressions
          \scompiler may produce a lot of output: \verb+>\dev\null+ is
          recommended to discard them;
			\item \verb+-smecy-accel+ asks for the generation of the
              accelerator parts, mainly by outlining the \texttt{map}-ped
              function;
			\item \verb+--smecy_lib=smecy_lib_directory/+ can be used to specify the path to the SMECY library; if specified it will be used instead of the environment variable \verb+SEMCY_LIB+;
			\item \verb+-std=c99+ should be used when compiling C99;
			\item \verb+-c+ will only translate input file instead of compiling it; with input file \verb+fileName.C+, \scompiler will generate a \verb+rose_fileName.C+ file with calls to SMECY API instead of SMECY pragmas;
			\item \verb+-fopenmp+ triggers OpenMP pragmas compilation using the back-end compiler.
		\end{itemize}

		\paragraph{Example} To compile a C99 input file with smecy and OpenMP pragmas without useless output type:
\begin{verbatim}
smecc -std=c99 -fopenmp -smecy input.c
\end{verbatim}

	\section{Known bugs and limitations}

		\paragraph{Features not yet implemented:}
		\begin{itemize}
			\item FORTRAN support;
			\item only toy implementation of the SMECY API.
		\end{itemize}

		\paragraph{AstRewriteMechanism bugs:}
		\begin{itemize}
			\item crash if the C++ input file has certain extensions (like ".cpp"), changing the extension to ".C" seems to solve the problem;
			\item the parser called for the strings is always in C++ mode (not C), commenting out a few lines in a ROSE header prevents front-end errors;
			\item the parsing of expressions is extremely slow (several seconds to parse ten expressions) and generates 1 file per expression to parse.
		\end{itemize}

		\paragraph{Compatibility with ROSE OpenMP lowering}
		\begin{itemize}
			\item ROSE OpenMP built-in support conflicts with smecy lowering and requires special handling;
			\item OpenMP files lowered using XOMP library require special linking, see in \verb+rose_install_dir/src/midend/programTransformation/ompLowering/+ for the library files.
		\end{itemize}

		\paragraph{Other bugs}
		\begin{itemize}
			\item if \verb+-smecy+ is not set, multi-line pragmas will lose their \verb+\+ and fail to compile.
		\end{itemize}

	\bibliography{biblio}
	\bibliographystyle{plain}

\end{document}


% Emacs religion:
%%% Local Variables:
%%% mode: latex
%%% ispell-local-dictionary: "american"
%%% TeX-PDF-mode: t
%%% TeX-master: t
%%% End:

% Be fair, vi religion too :-)
% vim: spell spelllang=en
